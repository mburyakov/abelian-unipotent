\documentclass[15pt]{article}
\usepackage[a5paper,margin=0.9in]{geometry}
\usepackage[utf8]{inputenc}
\usepackage[T2A]{fontenc}
\usepackage{hyperref}
\usepackage{amsmath}
\usepackage{amssymb}
\usepackage{amsthm}
\usepackage{mathtools}
\usepackage{extarrows}
\renewcommand{\proofname}{Доказательство}
\usepackage{amssymb}
\usepackage{tikz}
\usepackage{tikz-cd}


\title{Неунимодулярные линейные группы с абелевым унипотентным радикалом}
\author{}
\date{}

\usepackage{natbib}
\usepackage{graphicx}
\usepackage[russian, english]{babel}

\theoremstyle{remark}
\newtheorem{prop}{Утверждение}
\newtheorem{thm}{Теорема}
\newtheorem{lm}{Лемма}
\newtheorem{example}{Пример}

\linespread{1.3}

\newcommand{\Z}{\mathbb{Z}}
\renewcommand{\C}{\mathbb{C}}
\renewcommand{\le}{\leqslant}
\renewcommand{\ge}{\geqslant}
\begin{document}

\maketitle

\sloppy

\section{Введение}

Пусть $G_\Z$ --- редуктивная групповая схема с приводимой системой корней $\Phi = \Phi_1 + \dots + \Phi_n$, где $\Phi_i$ --- (приведённые) неприводимые системы корней.

Далее пусть $U_\Z$ --- неприводимое рациональное представление $G_\Z$, $\pi:G_\Z \to GL(U_\Z)$. Тогда $U=U_\Z\otimes R$ будет представлением группы $G=G_\Z(R)$ над кольцом $R$. Так как представление рационально, то $U$, как и $U_\Z$, будет являться свободным модулем, и в случае выбора максимального тора $T$ является (\#источник?) прямой суммой весовых подпространств
$$U=\bigoplus_{\sigma \in \Sigma \subset X(T)} {U \cap U_\sigma} ,$$
где\\
$U_\sigma = \{x \in U_\C | h x = \sigma(h) x \forall h \in T \}$,\\
$\Sigma = \{\sigma \in X(T) | U_\sigma \ne \varnothing\}$ --- множество весов представления,\\
$X(T)$ --- группа рациональных характеров тора.

Будем предполагать обратимость всех возникающих коэффициентов в коммутационных соотношениях Шевалле: $2,3 \in R^* $.
(\#где используется?).

Аддитивную группу $R[G]$-модуля $U$ можно рассматривать как абелеву группу с действием $G$ на ней, поэтому
имеет смысл полупрямое произведение $P = G \rightthreetimes U$.
%При этом условие неприводимости будет означать отсутствие в $U$ подгрупп, нормализуемых $G$.
Тогда $U$ как представление $G$ можно рассматривать, как частью присоединённого представления $P$. Это делает логичным использование мультипликативной нотации для обозначения сложения в $U$, а также обозначение действия $G$ на $U$ через $^{g}u = g u g^{-1}$. 

Выберем лежащий в весовых подпространствах базис $u_\sigma \in U_\sigma$. Разложение по базису будем записывать в виде $u = \prod_{\sigma \in \Sigma} u_\sigma(\xi_\sigma)$, где $\xi_\sigma$ --- коэффициенты разложения.

Тогда на весовых подпространствах корневые элементы группы действуют как
\begin{equation}\label{eq:commutator_relation}
[x_\alpha(\xi), u_\sigma(\eta)] = \prod_{i=1}^{d_\alpha} u_{\sigma+i\alpha}(N_{\sigma\alpha i}\,\xi^i \eta),
\end{equation}
где $N_{\sigma\alpha i}$ ---  некоторые целочисленные структурные константы. Выбор положительных корней $\Phi^+$ задаёт частичный порядок $\prec$ в $\Sigma$, причём неприводимость представления влечёт существование единственного старшего (относительно $\prec$) веса $\widetilde{\sigma}$, то есть $\forall \sigma \in \Sigma \  \exists \alpha \in \Phi : \sigma+\alpha = \widetilde{\sigma}$.

% в схемном смысле, рациональное представление - то есть сразу существует весовое разложение).

%Пусть $R$ --- коммутативное кольцо, $G = G(\Phi, R)$ --- редуктивная группа над $R$ с системой корней $\Phi = \Phi_1 + \dots + \Phi_n$.
%Пусть $U$ --- неприводимое представление $G$ над $R$ (\# в схемном смысле, рациональное представление - то есть сразу существует весовое разложение).

%Модуль Вейля V(pi)
%Неприводимые L(pi)
%Нужны модули вейля
%w(po) - экстремальные веса
%Микровесовое представление


\begin{example}
Пусть $ G = GL(2,\C) $, а $U$ --- её четырёхмерное представление, такое что
\begin{align*}
\left(
\begin{matrix}
1 & \xi \\ 0 & 1
\end{matrix}
\right)
& \mapsto
\left(
\begin{matrix}
1 & \xi & \xi^2 & \xi^3 \\
0 & 1   & 2\xi & 3\xi^2 \\
0 & 0   & 1     & 3\xi \\
0 & 0 & 0 & 1
\end{matrix}
\right)
\\
\left(
\begin{matrix}
1 & 0 \\ \xi & 1
\end{matrix}
\right)
& \mapsto
\left(
\begin{matrix}
1 & 0 & 0 & 0 \\
3\xi & 1   & 0 & 0 \\
3\xi^2 & 2\xi   & 1     & 0 \\
\xi^3 & \xi^2  & \xi & 1
\end{matrix}
\right)
\\
\left(
\begin{matrix}
t_1 & 0 \\ 0 & t_2
\end{matrix}
\right)
& \mapsto
\left(
\begin{matrix}
t_1^3 & 0 & 0 & 0 \\
0 & t_1^2 t_2 & 0 & 0 \\
0 & 0 & t_1 t_2^2 & 0 \\
0 & 0 & 0 & t_2^3
\end{matrix}
\right)
\end{align*}

Тогда весами этого представления будут характеры:
\begin{align*}
(t_1,t_2) & \mapsto t_1^3\\
(t_1,t_2) & \mapsto t_1^2 t_2\\
(t_1,t_2) & \mapsto t_1 t_2^2\\
(t_1,t_2) & \mapsto t_2^3
\end{align*}

\begin{center}
\begin{tikzpicture}[thick, scale=0.5]

\foreach \k in {1,...,4} {
  \draw[blue,dashed] (0,0) -- +(\k * 90:4);
}
\begin{scope}
\node[circle,inner sep=2,fill=red] at (0, 3) {};
\node[circle,inner sep=2,draw=red] at (1, 2) {};
\node[circle,inner sep=2,fill=red] at (2, 1) {};
\node[circle,inner sep=2,draw=red] at (3, 0) {};
\node at (3,-0.7) {3};
\node at (-0.7,3) {3};
\end{scope}

\end{tikzpicture}
\end{center}


Веса присоединённого представления $G$ выглядят так:

\begin{align*}
(t_1,t_2) & \mapsto 1\\
(t_1,t_2) & \mapsto t_1 t_2^{-1}\\
(t_1,t_2) & \mapsto t_1^{-1} t_2
\end{align*}

\begin{center}
\begin{tikzpicture}[thick, scale=0.5]

\foreach \k in {1,...,4} {
  \draw[blue,dashed] (0,0) -- +(\k * 90:4);
}
\begin{scope}
\node[circle,inner sep=2,fill=red] at (0, 3) {};
\node[circle,inner sep=2,draw=red] at (1, 2) {};
\node[circle,inner sep=2,fill=red] at (2, 1) {};
\node[circle,inner sep=2,draw=red] at (3, 0) {};
\node at (3,-0.7) {3};
\node at (-0.7,3) {3};
\end{scope}

\end{tikzpicture}
\end{center}

Веса действия G на $P = G \rightthreetimes U$ будут совокупностью приведённых выше весов:

\begin{center}
\begin{tikzpicture}[thick, scale=0.5]

\foreach \k in {1,...,4} {
  \draw[blue,dashed] (0,0) -- +(\k * 90:4);
}
\begin{scope}
\node[circle,inner sep=2,fill=blue] at (0, 0) {};
\node[circle,inner sep=2,draw=blue] at (-1, 1) {};
\node[anchor=east] at (-0.8,0.5) {$-\alpha_{1}$};
\node[circle,inner sep=2,draw=blue] at (1, -1) {};
\node[anchor=east] at (1.2,-1.5) {$\alpha_{1}$};

\node[circle,inner sep=2,fill=red] at (0, 3) {};
\node[anchor=west] at (0.1,3.5) {$\widetilde\sigma-3\alpha_{1}$};
\node[circle,inner sep=2,draw=red] at (1, 2) {};
\node[anchor=west] at (1.1,2.5) {$\widetilde\sigma-2\alpha_{1}$};
\node[circle,inner sep=2,fill=red] at (2, 1) {};
\node[anchor=west] at (2.1,1.5) {$\widetilde\sigma-\alpha_{1}$};
\node[circle,inner sep=2,draw=red] at (3, 0) {};
\node[anchor=west] at (3.1,0.5) {$\widetilde\sigma$};

\end{scope}
%\draw[->] (3,0) -- (2,1) node[below] {\(\alpha_1\)};

\end{tikzpicture}
\end{center}

\end{example}

Коммутационные соотношения (\ref{eq:commutator_relation}) будут выглядеть так:
\begin{align*}
[x_{\alpha_1}(\xi), u_{\widetilde\sigma}(\eta)] & = 1\\
[x_{\alpha_1}(\xi), u_{\widetilde\sigma-\alpha_1}(\eta)] & = u_{\widetilde\sigma}(\xi\eta)\\
[x_{\alpha_1}(\xi), u_{\widetilde\sigma-2\alpha_1}(\eta)] & = u_{\widetilde\sigma-\alpha_1}(2\xi\eta)+u_{\widetilde\sigma}(\xi^2\eta)\\
[x_{\alpha_1}(\xi), u_{\widetilde\sigma-3\alpha_1}(\eta)] & = u_{\widetilde\sigma-2\alpha_1}(3\xi\eta)+u_{\widetilde\sigma-\alpha_1}(3\xi^2\eta)+u_{\widetilde\sigma}(\xi^3\eta)
\end{align*}

\section{Элементарная группа}

Обозначим за $E = E(\Phi,R)$ подгруппу в $G$, порождённую корневыми унипотентами $x_\alpha(\xi)$, где $\alpha \in \Phi$, $\xi \in R$.

Сформулируем несколько простых свойств коммутаторов.

\begin{prop}
  $$[[g,u],v] = 0, \quad g \in G, \ u,v \in U $$
\end{prop}
\begin{proof}
  Очевидно следует из абелевости $U$.
  $$ [[g,u],v] = (g u g^{-1} u^{-1}) v (u g u^{-1} g^{-1}) v^{-1} = (gug^{-1}) (u^{-1}) (v) (u) (g^{-1}u^{-1}g) (v^{-1}) = 0$$
\end{proof}

\begin{prop}
  $$[g,uv] = [g,u][g,v], \quad g \in G, \ u,v \in U $$
\end{prop}

\begin{prop}
  $$[x_\alpha(-\xi),u] = [x_\alpha(\xi),u]^{-1}$$
\end{prop}
%\begin{proof}
  
%\end{proof}


\begin{prop}
  $$[x_\alpha(\xi),u]^n = [x_\alpha(\xi),u^n] = [x_\alpha(n\xi),u], \quad n \in \Z$$
\end{prop}

%Будем предполагать также, что $E$ действует на $U$ неприводимо, то есть в $U$ нет подгрупп, нормализуемых $E$.

%Рассмотрим семейства операторов $T_{\alpha,\xi} = [x_\alpha(\xi),\,\boldsymbol{\cdot}\,]$, где $\alpha \in \Phi$. Коядра и образы этих операторов имеют размерность не более $1$.

%Заметим, что если $\bigcup_{\xi\in R}\mathrm{Im}T_{\alpha_1,\xi} \notin \bigcap_{\xi\in R}\mathrm{Ker}T_{\beta,\xi}$ и $\bigcup_{\xi\in R}\mathrm{Im}T_{\alpha_2,\xi} \notin \bigcap_{\xi\in R}\mathrm{Ker}T_{\beta,\xi}$, то $\bigcup_{\xi\in R}\mathrm{Im}T_{\alpha_1,\xi} = \bigcup_{\xi\in R}\mathrm{Im}T_{\alpha_2,\xi}$.

%Поэтому можно рассмотреть граф, вершинами которого будут образы $s(\alpha) = \bigcup_{\xi\in R}\mathrm{Im}T_{\alpha,\xi}$ этих операторов, а рёбрами --- корни $\alpha \in \Phi$. В силу неприводимости действия $E$ на $U$ этот граф будет связным.

%Выбор положительных корней $\Phi^+ \subset \Phi$ задаёт ориентацию на графе, что, вследствие отсутствия циклов, влечёт частичный порядок на его вершинах.

%Обозначим множество вершин этого графа через $\Sigma$ и будем называть его элементы дополнительными корнями. Также в каждом одномерном подмодуле $\alpha \in \Sigma$ зафиксируем некоторый порождающий элемент $u_\alpha(1)$. Порождаемые им элементы $u_\alpha(1)^\xi$, $\xi \in R$, будем обозначать через $u_\alpha(\xi)$.

%Выпишем некоторые свойства $u_\alpha(\xi)$, делающие их похожими на корневые унипоненты:

\begin{prop}
  $$[x_\alpha(\theta\xi),u_\sigma(\eta)] = [x_\alpha(\xi),u_\sigma(\theta\eta)], \quad \theta \in R, \sigma \in \Sigma$$
\end{prop}

\begin{prop}
  $$ u_\sigma(\xi) u_\sigma(\eta) = u_\sigma(\xi + \eta), \quad \sigma \in \Sigma $$
\end{prop}

\begin{prop}
  В $\Sigma$ существует единственный максимальный корень $\widetilde{\sigma}$ и единственный минимальный корень.
\end{prop}

Таким образом, в $P = G \rightthreetimes U $ существует группа $E \rightthreetimes U$, которая порождается корнями $x_\alpha(\xi) \in E$ и весами $u_\sigma(\xi) \in U$.

\section{Подгруппы, нормализуемые $E$}

\begin{lm}(\citep{Stavrova2009}, Theorem 2.3, Corollary 2.4)
  \label{directproduct}
  Пусть $G = G(\Phi, R)$ --- редуктивная групповая схема Шевалле-Демазюра
  с системой корней $\Phi = \Phi_1 + \ldots + \Phi_n$, где каждая неприводимая система корней $\Phi_i$ имеет ранг не меньше $2$. В коммутативном кольце $R$ предполагается обратимость всех необходимых структурных констант.
  
  Тогда если подгруппа $H \le G$ нормализуется элементарной группой $E = E(\Phi,R)$, то её коммутант с $E$ можно записать в виде прямого произведения
  $$ [H, E] = \prod_{i=1}^n E(\Phi_i,R,I_i), $$
  где $E(\Phi_i,R,I_i) = E(\Phi_i,I_i)^{E(\Phi_i,R)}$, $I_i \trianglelefteq R$
\end{lm}

\begin{lm}(\citep{Stavrova2009}, Lemma 4.2)
  \label{transitivity}
  Для любых двух корней $\alpha, \beta \in \Phi$, таких что их сумма также является корнем, и для любых  $\xi \in R$, $I \trianglelefteq R$ выполнено
  $$ \left< x_\alpha(\xi) \right>^{X_\beta(I)} \ge X_{\alpha + \beta}(\xi I), $$  
  где $X_\alpha(I) = \{x_\alpha(\xi) | \xi \in I\}$ --- относительная корневая подгруппа в $G$.
\end{lm}

\begin{lm}\label{unipotenttransitivity}
  Для любых двух весов $\sigma, \tau \in \Sigma$ элементы $u_\sigma(\xi)$ и $u_\tau(\xi)$ сопряжены относительно действия $E(\Phi,R)$, сиречь  
  $$ u_\tau(\xi) \in \left<u_\sigma(\xi)\right>^{E(\Phi,R)}. $$
\end{lm}

\begin{thm}
  Снова $G = G(\Phi, R)$ --- редуктивная групповая схема Шевалле-Демазюра
  с системой корней $\Phi = \Phi_1 + \ldots + \Phi_n$, где каждая $\Phi_i$ --- неприводимая система корней ранга не меньше $2$. $U$ --- неприводимое рациональное представление $G$. В коммутативном кольце $R$ предполагается обратимость всех структурных констант группы $G$, а также констант $N_{\sigma\alpha i}$ из соотношений (\ref{eq:commutator_relation}).
    
  Пусть имеется подгруппа $H \le G \rightthreetimes U$, нормализуемая группой $E$, то есть $[H,E] \le H$. Тогда образ $H$ при проекции $G \rightthreetimes U \rightarrow G$, обозначаемый как $H_G$, будет обладать следующим свойством: $[H_G,E]\le H$.
\end{thm}
\begin{proof}
  $H$ нормализуется $E$, следовательно $H_G$ также нормализуется $E$. По лемме \ref{directproduct} $[H_G,E] = \prod_{i=1}^n E(\Phi_i,R,I_i)$ при некотором выборе идеалов $I_i$. Требуется доказать, что $[H_G,E] \le H$, то есть в прямом произведении $\prod_{i=1}^n E(\Phi_i,R,I_i)$ каждый множитель $E(\Phi_i,R,I_i)$  лежит в $H$.
  
\begin{lm}
  $$X_{\widetilde{\beta_i}}(I_i)^{E(\Phi_i,R)} = E(\Phi_i,R,I_i),$$
  где $\widetilde{\beta}_i$ --- максимальный корень в $\Phi_i$.
\end{lm}
\begin{proof}
  Очевидно, что
\begin{align*}
  X_{\widetilde{\beta_i}}(I_i) &\le E(\Phi_i,I_i) \\
  X_{\widetilde{\beta_i}}(I_i)^{E(\Phi_i,R)} &\le E(\Phi_i,I_i)^{E(\Phi_i,R)} = E(\Phi_i,R,I_i)
\end{align*}
  Обратное включение вытекает из леммы \ref{transitivity}.
  
  Действительно, возьмём в лемме \ref{transitivity} в качестве $\alpha$ максимальный корень $\widetilde{\beta_i}$, а в качестве идеала $I$ всё кольцо $R$. Тогда
  $$ \left< x_{\widetilde{\beta_i}}(\xi) \right>^{X_\beta(R)} \ge X_{\widetilde{\beta_i} + \beta}(\xi R). $$
  Но так как любой корень из $\Phi_i$ может быть получен прибавлением к максимальному корню $\widetilde{\beta_i}$ некоторого отрицательного корня $\beta \in \Phi^-$, то группа $E(\Phi_i,I_i)$ порождается подгруппами $X_{\widetilde{\beta_i} + \beta}(I_i)$, а следовательно 
\begin{align*}
E(\Phi_i,I_i) &\le \left< X_{\widetilde{\beta_i}}(I) \right>^{E(\Phi_i,R)}\\
  E(\Phi_i,R,I_i) = E(\Phi_i,I_i)^{E(\Phi_i,R)} &\le \left< X_{\widetilde{\beta_i}}(I) \right>^{E(\Phi_i,R)}
\end{align*}
\end{proof}

Таким образом, необходимо доказать, что $X_{\widetilde{\beta_i}}(I_i)^{E(\Phi_i,R)} \le H$, то есть что
$x_{\widetilde{\beta_i}}(\xi) \in H \ \forall \xi \in I_i$.

По построению известно, что $x_{\widetilde{\beta_i}}(\xi) \in [H_G,E] \in H_G$.
Обозначим за $h$ некоторый прообраз $x_{\widetilde{\beta_i}}(\xi) \in [H_G,E] \in H_G$ при проекции $G \rightthreetimes U \rightarrow G$.

\begin{equation*}
\tikzset{
  Subgroup/.style={
    draw=none,
    every to/.append style={
      edge node={node [sloped, allow upside down, auto=false]{$\le$}}}},
  Equals/.style={
    draw=none,
    every to/.append style={
      edge node={node [sloped, allow upside down, auto=false]{$=$}}}},
  Included/.style={
    draw=none,
    every to/.append style={
      edge node={node [sloped, allow upside down, auto=false]{$\in$}}}}
}
\begin{tikzcd}
G \rightthreetimes U \arrow{r}{} & G \\
H \arrow[Subgroup]{u} \arrow{r}{} & H_G \arrow[Subgroup]{u} \\
h \arrow[Included]{u} \arrow[maps to]{r}{} & x_{\widetilde{\beta_i}} \arrow[Included]{u} \\
x_{\widetilde{\beta_i}}\,u \arrow[Equals]{u} \arrow[maps to]{r}{} & x_{\widetilde{\beta_i}} \arrow[Equals]{u} \\
\end{tikzcd}
\end{equation*}

Отсюда видно, что $x_{\widetilde{\beta_i}}(\xi)$ можно представить в виде произведения $$x_{\widetilde{\beta_i}}(\xi) = h u^{-1} = h \prod_{\sigma \in \Sigma} u_\sigma(\theta_\sigma),$$

где $h\in H\cap G$, $u \in U$, а $\theta_\sigma \in R$.

Докажем, что если в произведении содержится (с ненулевым $\theta_\sigma$, иначе $u_\sigma(0)=1$) хотя бы один множитель $x_\sigma(\theta_\sigma)$, для которого $\sigma \ne \widetilde\sigma$ (например, если множителей несколько), то $u \in H$.%можно найти другое разложение $x_{\widetilde{\beta_i}}(\xi) = h' u'^{-1}$, в котором $u'^{-1}$ раскладывается на меньшее число множителей (при этом уже не обязательно $h' \in G$, а только лишь $h' \in H$ и $u' \in U$).


Пусть $u_\sigma(\theta_\sigma)$ --- множитель с минимальным (относительно $\prec$) из присутствующих в произведении весов, и $\sigma$ не является максимальным весом $\widetilde{\sigma}$. Тогда найдётся $\beta \in \Phi_i^+$ (индекс $i$ будем в дальнейшем опускать), такой что $\beta+\sigma \in \Sigma$. Тогда

$$
w \coloneqq [x_\beta(1), x_{\widetilde{\beta}}(\xi)u] =
x_\beta(1) x_{\widetilde{\beta_i}}(\xi) u x_\beta(-1) u^{-1} x_{\widetilde{\beta_i}}(-\xi)
\xlongequal{\beta+\widetilde{\beta_i}\notin\Phi_i}$$
$$=x_\beta(1) x_{\widetilde{\beta_i}}(\xi) u x_\beta(-1) u^{-1} x_{\widetilde{\beta_i}}(-\xi) = \,
^{x_{\widetilde{\beta_i}}(\xi)}[x_\beta(1),u] \xlongequal{\text{similar}} [x_\beta(1),u] \in U
$$ 

Но также $ w = [x_\beta(1), x_{\widetilde{\beta_i}}(\xi)u] \in H$, так как $x_{\widetilde{\beta_i}}(\xi)u \in H$. При этом $w \ne 1$, так как
\begin{align*}
w = [x_\beta(1),u] = [x_\beta(1),u_\sigma(\theta_\sigma)\prod_{\tau\succ\sigma}u_\tau(\theta_\tau)] =\\
=[x_\beta(1),u_\sigma(\theta_\sigma)][x_\beta(1),\prod_{\tau\succ\sigma}u_\tau(\theta_\tau)]=
u_{\sigma+\beta}(\theta_\sigma)\cdot \;\ldots \ne 1.
\end{align*}

Но из леммы \ref{unipotenttransitivity} вытекает, что
$u \in \left<w\right>^{E(\Phi_i,R)}$, поэтому $u \in \left<H\right>^{E(\Phi_i,R)} = H$.

Осталось доказать случай, когда остался последний множитель $x_{\widetilde{\sigma}}(\theta_{\widetilde{\sigma}})$. Его доказательство является предметом дальнейших исследований.


\end{proof}

\bibliographystyle{plain}
\bibliography{references}
\end{document}
