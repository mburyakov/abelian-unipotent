\documentclass[15pt]{article}
%\usepackage[a5paper,margin=0.5in]{geometry}
\usepackage[utf8]{inputenc}
\usepackage[T2A]{fontenc}
\usepackage{amsmath}
\usepackage{amsthm}
\usepackage{mathtools}
\usepackage{extarrows}
\renewcommand{\proofname}{Доказательство}
\usepackage{amssymb}


\title{Неунимодулярные линейные группы с абелевым унипотентным радикалом}
\author{}
\date{}

\usepackage{natbib}
\usepackage{graphicx}
\usepackage[russian, english]{babel}

\theoremstyle{remark}
\newtheorem{prop}{Утверждение}
\newtheorem{thm}{Теорема}
\newtheorem{lm}{Лемма}

\linespread{1.3}

\begin{document}

\maketitle

\sloppy

\section{Введение}

Пусть $R$ --- коммутативное кольцо, $G = G(\Phi, R)$ --- редуктивная группа над $R$ с системой корней $\Phi = \Phi_1 + \dots + \Phi_n$.
Пусть $U$ --- неприводимое представление $G$ над $R$ (\#с каким-то ещё условием, видимо надо брать неприводимое).
Будем предполагать обратимость всех возникающих коэффициентов в коммутационных соотношениях Шевалле: $2,3 \in R^* $.
(\#Все не нужны, но не забыть про те, которые возникнут из $U$).

$R[G]$-модуль $U$ можно рассматривать как аддитивную группу с действием $G$ на ней, поэтому
имеет смысл полупрямое произведение $G \rightthreetimes U$.
При этом условие неприводимости будет означать отсутствие в $U$ подгрупп, нормализуемых $G$.
Действие $G$ на $U$ будем обозначать как $^{g}u = g u g^{-1}$.

\section{Элементарная группа}

\begin{prop}
  $$[[g,u],v] = 0$$
\end{prop}
\begin{proof}
  Очевидно следует из абелевости $U$.
  $$ [[g,u],v] = (g u g^{-1} u^{-1}) v (u g u^{-1} g^{-1}) v^{-1} = (gug^{-1}) (u^{-1}) (v) (u) (g^{-1}u^{-1}g) (v^{-1}) = 0$$
\end{proof}

Обозначим за $E = E(\Phi,R)$ подгруппу в $G$, порождённую корневыми унипотентами $x_\alpha(\xi)$, где $\alpha \in \Phi$, $\xi \in R$.

Заметим, что
\begin{prop}
  $$[x_\alpha(-\xi),u] = [x_\alpha(\xi),u]^{-1}$$
\end{prop}
\begin{proof}
  
\end{proof}


\begin{prop}
  $$[x_\alpha(\theta\xi),u] = [x_\alpha(\xi),u]^\theta, \, \theta \in R$$
\end{prop}
\begin{prop}
  $$[x_\alpha(\xi),u^\theta] = [x_\alpha(\xi),u]^\theta, \, \theta \in R$$
\end{prop}
\begin{prop}
  $$[x_\alpha(\xi),[x_\alpha(\xi),u]] = 1$$
\end{prop}

Будем предполагать также, что $E$ действует на $U$ неприводимо, то есть в $U$ нет подгрупп, нормализуемых $E$.

Рассмотрим семейства операторов $T_{\alpha,\xi} = [x_\alpha(\xi),\,\boldsymbol{\cdot}\,]$, где $\alpha \in \Phi$. Коядра и образы этих операторов имеют размерность не более $1$.

Заметим, что если $\bigcup_{\xi\in R}\mathrm{Im}T_{\alpha_1,\xi} \notin \bigcap_{\xi\in R}\mathrm{Ker}T_{\beta,\xi}$ и $\bigcup_{\xi\in R}\mathrm{Im}T_{\alpha_2,\xi} \notin \bigcap_{\xi\in R}\mathrm{Ker}T_{\beta,\xi}$, то $\bigcup_{\xi\in R}\mathrm{Im}T_{\alpha_1,\xi} = \bigcup_{\xi\in R}\mathrm{Im}T_{\alpha_2,\xi}$.

Поэтому можно рассмотреть граф, вершинами которого будут образы $s(\alpha) = \bigcup_{\xi\in R}\mathrm{Im}T_{\alpha,\xi}$ этих операторов, а рёбрами --- корни $\alpha \in \Phi$. В силу неприводимости действия $E$ на $U$ этот граф будет связным.

Выбор положительных корней $\Phi^+ \subset \Phi$ задаёт ориентацию на графе, что, вследствие отсутствия циклов, влечёт частичный порядок на его вершинах.

Обозначим множество вершин этого графа через $\Sigma$ и будем называть его элементы дополнительными корнями. Также в каждом одномерном подмодуле $\alpha \in \Sigma$ зафиксируем некоторый порождающий элемент $u_\alpha(1)$. Порождаемые им элементы $u_\alpha(1)^\xi$, $\xi \in R$, будем обозначать через $u_\alpha(\xi)$.

Выпишем некоторые свойства $u_\alpha(\xi)$, делающие их похожими на корневые унипоненты:

\begin{prop}
  $$ u_\sigma(\xi) u_\sigma(\eta) = u_\sigma(\xi + \eta), \; \sigma \in \Sigma $$
  $$ [u_\sigma(\xi), u_\tau(\eta)] = 1, \; \sigma,\tau \in \Sigma $$
  $$ [x_\alpha(\xi), u_\sigma(\eta)] = u_{s(\alpha)}(c_{\alpha\sigma}\xi\eta), \; \alpha \in \Phi, \sigma \in \Sigma,$$ если ребро $\alpha$ имеет своим началом корень $\sigma$, иначе $ [x_\alpha(\xi), u_\sigma(\eta)]=1 $.
\end{prop}

В дальнейшем будет обозначать $s(\alpha)$ как $\alpha+\sigma \in \Sigma$, если ребро $\alpha$ имеет своим началом корень $\sigma$, иначе будем говорить, что  $\alpha+\sigma \notin \Sigma$.

\begin{prop}
  В $\Sigma$ существует единственный максимальный корень $\widetilde{\sigma}$.
\end{prop}

\begin{prop}
  В $\Sigma$ существует единственный минимальный корень $r$.
\end{prop}

Таким образом, мы построили дополненную элементарную группу $E \rightthreetimes U \le G \rightthreetimes U$, порождаемую корнями $x_\alpha(\xi) \in E$ и дополнительными корнями $x_\sigma(\xi) \in U$.

\section{Подгруппы, нормализуемые $E$}

\begin{lm}(\citep{Stavrova2009}, Theorem 2.3, Corollary 2.4)
  \label{directproduct}
  Пусть $G = G(\Phi, R)$ --- редуктивная групповая схема Шевалле-Демазюра
  с системой корней $\Phi = \Phi_1 + \ldots + \Phi_n$, где каждая неприводимая система корней $\Phi_i$ имеет ранг не меньше $2$. В коммутативном кольце $R$ предполагается обратимость всех необходимых структурных констант.
  
  Тогда если подгруппа $H \le G$ нормализуется элементарной группой $E = E(\Phi,R)$, то её коммутант с $E$ можно записать в виде прямого произведения
  $$ [H, E] = \prod_{i=1}^n E(\Phi_i,R,I_i), $$
  где $E(\Phi_i,R,I_i) = E(\Phi_i,I_i)^{E(\Psi,R)}$, $I_i \trianglelefteq R$
\end{lm}

\begin{lm}(\citep{Stavrova2009}, Lemma 4.2)
  \label{transitivity}
  Для любых двух корней $\alpha, \beta \in \Phi$ что их сумма также лежит в $\Phi$, и для любых  $\xi \in R$, $I \trianglelefteq R$ выполнено
  $$ \left< x_\alpha(\xi) \right>^{X_\beta(I)} \ge X_{\alpha + \beta}(\xi I) $$  
\end{lm}


\begin{thm}
  Снова $G = G(\Phi, R)$ --- редуктивная групповая схема Шевалле-Демазюра
  с системой корней $\Phi = \Phi_1 + \ldots + \Phi_n$, где каждая $\Phi_i$ --- неприводимая система корней ранга не меньше $2$. В коммутативном кольце $R$ предполагается обратимость всех необходимых структурных констант. $U$ --- неприводимое представление $G$, такое что $E$ действует на нём также неприводимо.
    
  Пусть имеется подгруппа $H \le G \rightthreetimes U$, нормализуемая группой $E$, то есть $[H,E] \le H$. Тогда образ $H$ при проекции $G \rightthreetimes U \rightarrow G$, обозначаемый как $H_G$, будет обладать следующим свойством: $[H_G,E]\le H$.
\end{thm}
\begin{proof}
  $H$ нормализуется $E$, следовательно $H_G$ также нормализуется $E$. По лемме \ref{directproduct} $[H,E] = \prod_{i=1}^n E(\Phi_i,R,I_i)$. Требуется доказать, что $[H,E] = \prod_{i=1}^n E(\Phi_i,R,I_i) \le H$, то есть $ E(\Phi_i,R,I_i) \le H \; \forall i \in 1 \ldots n $.
  
\begin{lm}
  $$X_{\widetilde{\beta_i}}(I_i)^{E(\Psi_i,R,I_i)} = E(\Psi_i,R,I_i),$$
  где $\widetilde{\beta}_i$ --- максимальный корень в $\Phi_i$.
\end{lm}
\begin{proof}
  Непосредственно вытекает из леммы \ref{transitivity}.
\end{proof}

Таким образом, необходимо доказать, что $x_{\widetilde{\beta_i}}(\xi) \le H \; \forall \xi \in R$.

По построению известно, что $x_{\widetilde{\beta_i}}(\xi) \in H_G$.
Разложим $x_{\widetilde{\beta_i}}(\xi)$ в произведение $$x_{\widetilde{\beta_i}}(\xi) = h u^{-1} = h \prod x_\sigma(u_\sigma),$$

где произведение берётся по дополнительным корням согласованно с частичным порядком на них.

Будем по очереди доказывать, что все множители так же лежат в $H$.

Пусть корень $\sigma$ в первом (минимальном) множителе не совпадает с $\widetilde{\sigma}$. Тогда найдётся $\beta \in \Phi^+$, такой что $\beta+\sigma \in \Sigma$. Тогда

$$
w \coloneqq [x_\beta(1), x_{\widetilde{\beta}}(\xi)u] =
x_\beta(1) x_{\widetilde{\beta}}(\xi) u x_\beta(-1) u^{-1} x_{\widetilde{\beta}}(-\xi)
\xlongequal{\beta+\widetilde{\beta}\notin\Sigma}$$
$$=x_\beta(1) x_{\widetilde{\beta}}(\xi) u x_\beta(-1) u^{-1} x_{\widetilde{\beta}}(-\xi) = \,
^{x_{\widetilde{\beta}}(\xi)}[x_\beta(1),u] \xlongequal{\text{similar}} [x_\beta(1),u] \in U
$$ 

Но также $ w \coloneqq [x_\beta(1), x_{\widetilde{\beta}}(\xi)u] \in H$, так как $x_{\widetilde{\beta}}(\xi)u \in H$. Следовательно, так как в U не может быть подгрупп, нормализуемых $E$, то и все множители в разложении $w = \prod x_\sigma(w_\sigma)$ лежат в $U$. А минимальный множитель в этом разложении имеет вид $x_{\sigma+\beta}(c u_\sigma)$, где $c \in R^*$. Поэтому и минимальный множитель в разложении $u = \prod x_\sigma(u_\sigma)$ также лежит в $U$.

Осталось доказать случай, когда остался последний множитель $x_{\widetilde{\sigma}}(u_{\widetilde{\sigma}})$. Его доказательство является предметом дальнейших исследований.


\end{proof}

\bibliographystyle{plain}
\bibliography{references}
\end{document}
