\documentclass[10pt]{article}
\usepackage[a5paper,margin=0.1in]{geometry}
\usepackage[utf8]{inputenc}
\usepackage[T2A]{fontenc}
\usepackage{natbib}
\usepackage{graphicx}
\usepackage[english, russian]{babel}
\usepackage{hyperref}
\usepackage{amsmath}
\usepackage{amssymb}
\usepackage{amsthm}
\usepackage{mathtools}
\usepackage{extarrows}
\usepackage{amssymb}
\usepackage{etoolbox}
\usepackage{tikz}
\usepackage{tikz-cd}
\usetikzlibrary{calc}
\usetikzlibrary{patterns}
\usetikzlibrary{decorations.pathreplacing}
\usepackage{xcolor}

\makeatletter
\newcounter{aaa}
\tikzset{
  apply/.style args={#1 except on segments #2}{postaction={
      /utils/exec={
        \@for\mattempa:=#2\do{\csdef{aaa@\mattempa}{}}
        \setcounter{aaa}{0}
      },
      decorate,decoration={show path construction,
        moveto code={},
        lineto code={
          \stepcounter{aaa}
          \ifcsdef{aaa@\theaaa}{}{
            \path[#1] (\tikzinputsegmentfirst) -- (\tikzinputsegmentlast);
          }
        },
        curveto code={
          \stepcounter{aaa}
          \ifcsdef{aaa@\theaaa}{}{
            \path [#1] (\tikzinputsegmentfirst) .. controls
            (\tikzinputsegmentsupporta) and (\tikzinputsegmentsupportb)
            ..(\tikzinputsegmentlast);
          }
        },
        closepath code={
          \stepcounter{aaa}
          \ifcsdef{aaa@\theaaa}{}{
            \path [#1] (\tikzinputsegmentfirst) -- (\tikzinputsegmentlast);
          }
        },
      },
    },
  },
}
\makeatother

\title{Неунимодулярные линейные группы с абелевым унипотентным радикалом}
\author{}
\date{}


\theoremstyle{remark}
\newtheorem{prop}{Утверждение}
\newtheorem{thm}{Теорема}
\newtheorem{lm}{Лемма}
\newtheorem{example}{Пример}
\newtheorem{definition}{Определение}


\linespread{1.3}

\newcommand{\Z}{\mathbb{Z}}
\newcommand{\N}{\mathbb{N}}
\renewcommand{\C}{\mathbb{C}}
\renewcommand{\le}{\leqslant}
\renewcommand{\ge}{\geqslant}
\begin{document}
\maketitle

%\sloppy

\section{Введение}


\begin{definition} (\cite{Conrad11reductivegroup})
Редуктивной $R$-группой называется линейная алгебраическая $R$-группа, не содержащая нетривиальных связных нормальных унипотентных линейных алгебраических $R$-подгрупп.
\end{definition}

\begin{definition}
Полупростой $R$-группой называется линейная алгебраическая $R$-группа, не содержащая нетривиальных связных нормальных разрешимых линейных алгебраических $R$-подгрупп.
\end{definition}

\begin{definition}
Простой $R$-группой называется линейная алгебраическая $R$-группа, не содержащая нетривиальных связных нормальных линейных алгебраических $R$-подгрупп.
\end{definition}

Пусть $G_\Z$ --- связная редуктивная расщепимая (\#где используется) групповая схема с системой корней $\Phi = \Phi_1 + \dots + \Phi_n$, где $\Phi_i$ --- (приведённые) неприводимые системы корней, $n \ge 1$.

Далее пусть $U_\Z$ --- неприводимое (над замкнутым полем характеристики $0$) рациональное представление (модуль Вейля) $G_\Z$, $\pi:G_\Z \to GL(U_\Z)$ с рангом
\begin{equation}\label{representationrank}\mathrm{rk} \; U_\Z \ge 2 \;. \end{equation}
Тогда $U=U_\Z\otimes R$ будет представлением группы $G=G_\Z(R)$ над кольцом $R$. Так как представление рационально, то $U$, как и $U_\Z$, будет являться свободным модулем, и в случае выбора максимального тора $T$ является (\citep{Borel1970}) прямой суммой весовых подпространств
$$U=\bigoplus_{\sigma \in \Sigma \subset X(T)} {U \cap U_\sigma} ,$$
где\\
$U_\sigma = \{x \in U_\C \; | \; ^h x = \sigma(h) x \forall h \in T \}$,\\
$\Sigma = \{\sigma \in X(T) \
; | \; U_\sigma \ne \varnothing\}$ --- множество весов представления,\\
$X(T)$ --- группа рациональных характеров тора.

\section{Основные сведения о группе $G$ и её рациональном представлении $\pi$}

$x_\alpha(\xi)$ --- корневые элементы группы $G$.

Их свойства:

\begin{prop}
Для экстремальных весов (в частности, для максимального веса $\widetilde\sigma$), весовые пространства $U_\sigma$ одномерны.
\end{prop}

\begin{prop}
Корневые элементы унипотентны (как операторы конечномерного представления $(U,\pi)$):
$$\exists n : (x_\alpha(\xi)-1)^n = 1$$
Для базисных представлений (микровесовые представления, представления на коротких корнях и присоединённые представления групп с сисмемами корней с простыми связями, \cite{Plotkin1998}) это верно уже при $n=3$. Для микровесовых представлений это верно уже при $n=2$.
\end{prop}

\begin{prop}
Логарифм является полиномиальной функцией на корневых элементах:
$$ \varepsilon_\alpha(\xi) \coloneqq \mathrm{ln} \, x_\alpha(\xi) = \sum_{i=0}^{n-1} {x_\alpha(\xi)^i k_i} $$
При этом коэффициенты полинома рациональны, а знаменатели в них не превышают $n-1$.
\end{prop}

\begin{prop} Аддитивность корневых элементов по параметру:
$$x_\alpha(\xi) \; x_\alpha(\theta)  = x_\alpha(\xi+\theta)$$
$$x_\alpha(\xi)^n = x_\alpha(n \, \xi) \; \forall n \in \Z$$
\end{prop}

\begin{prop}
$$ \varepsilon_\alpha(n \, \xi) = \mathrm{ln} \, x_\alpha(n \, \xi) = \mathrm{ln} \, x_\alpha(\xi)^n = n \; \mathrm{ln} \, x_\alpha(\xi) = n \; \varepsilon_\alpha(\xi) \; \forall n \in \Z
$$
\end{prop}

Если при этом константы выносятся с некоторым автоморфизмом поля (например, с комплексным сопряжением), то можно рассмотреть изоморфный исходному модуль Вейля, в котором
\begin{equation} \label{loglinearity}
\varepsilon_\alpha(\eta \, \xi) = \eta \; \varepsilon_\alpha(\xi) \; \forall \eta \in R
\end{equation}
Поэтому не умаляя общности (\#как строго сформулировать?) будем в дальнейшем предполагать \ref{loglinearity}.

\begin{prop}
Оператор $\varepsilon_\alpha(\xi)$ переводит элементы из подпространства $U_\sigma$ в подпространство $U_{\sigma+\alpha}$.
\end{prop}
\begin{proof}
\begin{multline*}
^h(\varepsilon_\alpha(\xi) \, u) =
^h\left(\sum_i \, {^{x_\alpha(\xi)^i}u \, k_i}\right) = 
\sum_i \, {^{h x_\alpha(\xi)^i}u \, k_i} = \\ =
\sum_i \, {^{(^h x_\alpha(\xi))^i h}u \, k_i} = 
\sum_i \, {^{(x_\alpha(\alpha(h)\cdot\xi))^i} \left(^h u\right) \, k_i} =\\=
\sum_i \, {^{(x_\alpha(\alpha(h)\cdot\xi))^i} u \cdot \sigma(h) \, k_i} =
\varepsilon_\alpha(\alpha(h)\cdot\xi) \, u \cdot \sigma(h) = \\ =
\varepsilon_\alpha(\xi) \, u \cdot \alpha(h)\sigma(h)
\end{multline*}
\end{proof}

\begin{prop} Результатом применения оператора $\varepsilon_\alpha(1)^n$ к ненулевому $u \in U_\sigma$ является ноль, только если $u \in \mathrm{Ker}\,\varepsilon_\alpha(1)$ или $\sigma + \alpha \notin \Sigma$.
\end{prop}
\begin{proof}
\# Это правда? Для микровесовых представлений утверждение тривиально.
\end{proof}

\begin{prop}
$U$ порождается (как модуль над полем хактеристики 0) элементами $\left(\prod_{\alpha}\varepsilon_\alpha(1)\right) \, u$, где $u$ --- некоторый произвольно выбранный вектор, а $\alpha$ пробегает по всевозможным последовательностям простых корней.
\end{prop}
\begin{proof}
\# Для полупростых групп следует из неприводимости представления, для редуктивных, видимо, тоже, так как радикал есть связная компонента центра, а фактор по радикалу --- полупростая группа.
\end{proof}

\begin{prop}
Для любого $u \in U_\sigma$ найдётся положительный корень $\alpha$, для которого $u \notin \mathrm{Ker} \, \varepsilon_\alpha(1)$
\end{prop}
\begin{proof}
\# Должно следовать из \ref{representationrank}.
\end{proof}

Рассмотрим ненулевой элемент $v = \varepsilon_\beta(1) \, u$, лежащий в $U_{\sigma+\beta}$. Исходный элемент $u$ может быть разложен по базису, состоящему из элементов вида $\left(\prod_{\alpha}\varepsilon_\alpha(1)\right) \, v$. Коэффициенты разложения рациональны (\textbf{\color{red}ЧУШЬ!!!}), поэтому можно рассмотреть из общий знаменатель $d$. В зависимости от конкретного выбора базиса (из бесконечного набора векторов вида $\left(\prod_{\alpha}\varepsilon_\alpha(1)\right) \, v$), а также от выбора исходного вектора $u$ и корня $\beta$ будут получаться различные целые $d$. Можно видеть, что множество всех возможных значений $d$, которые можно получить таким образом, замкнуто относительно умножения, то есть вместе с единицей (если она не была включена в него ранее) образует мультипликативное множество $S \subset \Z$. В дальнейшем будем рассматривать представление $U$ только над такими кольцами $R$, в которых
\begin{equation}\label{representationconstantinvertibility} S\subset R^* \; .\end{equation}

\begin{prop}\label{weightprojections}
Если $V$ --- инвариантный подмодуль $U$, то $V = \bigoplus V_\sigma$, то есть проекции вектора $v \in V$ на корневые подпространства также лежат в $V$.
\end{prop}
\begin{proof}
Пусть $v = \sum v_\sigma$. Доказательство проведём индукцией по минимальному весу $\sigma'$ из присутствующих в сумме. Базой индукции будет сумма из одного слагаемого старшего веса: $v = v_{\widetilde\sigma}$.

Пусть теперь $v = v_{\sigma'} + v'$, где $v'$, если лежит в $V$, то раскладывается в $V$ по весовым простраствам по индукционному предположению. Тогда найдётся такая последовательность положительных корней, что $w := \left(\prod_{\alpha}\varepsilon_\alpha(1)\right) \, v \in U_{\widetilde\sigma}$. Но одновременно $w = \left(\prod_{\alpha}\varepsilon_\alpha(1)\right) \, v_{\sigma'}$, поэтому, в силу \ref{representationconstantinvertibility}, $v_{\sigma'}$ также лежит в $V$.
\end{proof}

\begin{prop}\label{unipotentsubgroups}
Над кольцами, для которых выполняется (\ref{representationconstantinvertibility}), инвариантные подмодули $U$ имеют вид $U_I = U \cdot I$, где $I \trianglelefteq R$.
\end{prop}
\begin{proof}
Пусть $V$ --- инвариантный подмодуль $U$. Достаточно доказать, что найдётся идеал $I$, такой что любой $V \cap U_\sigma = U_\sigma \, I$.

Пусть $v$ --- произвольно выбранный элемент в $V \cap U_\sigma$. Тогда найдётся такая последовательность положительных корней, что $w := \left(\prod_{\alpha}\varepsilon_\alpha(1)\right) \, v \in V \cap U_{\widetilde\sigma}$. В силу одномерности $U_{\widetilde\sigma}$ подмодуль $W := V \cap U_{\widetilde\sigma} \subseteq U_{\widetilde\sigma}$, состоящий из таких $w$, будет иметь вид $U_{\widetilde\sigma} \, I$ для некоторого идеала $I$. В силу \ref{representationconstantinvertibility}, если $w = \widetilde u \, \xi$, то $v = u \, \xi$.
\end{proof}


\section{Что-то дальше}

Аддитивную группу $R[G]$-модуля $U$ можно рассматривать как абелеву группу с действием $G$ на ней, поэтому
имеет смысл полупрямое произведение $P = G \rightthreetimes U$.
Тогда $U$ как представление $G$ можно рассматривать как часть присоединённого представления $P$. Это делает логичным использование мультипликативной нотации для обозначения сложения в $U$, а также обозначение действия $G$ на $U$ через $^{g}u = g u g^{-1}$. 

\section{Элементарная группа}

Обозначим за $E = E(\Phi,R)$ подгруппу в $G$, порождённую корневыми унипотентами $x_\alpha(\xi)$, где $\alpha \in \Phi$, $\xi \in R$.

Сформулируем несколько простых свойств коммутаторов в $G \rightthreetimes U$.

\begin{prop}
  $$[[g,u],v] = 1, \quad g \in G, \ u,v \in U $$
\end{prop}
\begin{proof}
  Очевидно следует из абелевости $U$.
\begin{align*}
[[g,u],v] = (g u g^{-1} u^{-1}) v (u g u^{-1} g^{-1}) v^{-1} = \\
= (gug^{-1}) (u^{-1}) (v) (u) (g^{-1}u^{-1}g) (v^{-1}) = 1
\end{align*}
\end{proof}

\begin{prop}
  $$[g,uv] = [g,u][g,v], \quad g \in G, \ u,v \in U $$
\end{prop}

\begin{prop}
  В $\Sigma$ существует единственный максимальный вес $\widetilde\sigma$ и единственный минимальный вес.
\end{prop}

Очевидно, в $P = G \rightthreetimes U $ существует подгруппа группа $E \rightthreetimes U$, которая порождается корневыми элементами $x_\alpha(\xi) \in E$ и элементами $u \in U$.

\section{Подгруппы, нормализуемые $E$}

\begin{lm}(\citep{Stavrova2009}, Theorem 2.3, Corollary 2.4)
  \label{directproduct}
  Пусть $G = G(\Phi, R)$ --- редуктивная групповая схема Шевалле-Демазюра
  с системой корней $\Phi = \Phi_1 + \ldots + \Phi_n$, где каждая неприводимая система корней $\Phi_i$ имеет ранг не меньше $2$. В коммутативном кольце $R$ предполагается обратимость всех необходимых структурных констант.
  
  Тогда если подгруппа $H \le G$ нормализуется элементарной группой $E = E(\Phi,R)$, то её коммутант с $E$ можно записать в виде прямого произведения
  $$ [H, E] = \prod_{i=1}^n E(\Phi_i,R,I_i), $$
  где $E(\Phi_i,R,I_i) = E(\Phi_i,I_i)^{E(\Phi_i,R)}$, $I_i \trianglelefteq R$
\end{lm}

\begin{lm}(\citep{Stavrova2009}, Lemma 4.2)
  \label{transitivity}
  Для любых двух корней $\alpha, \beta \in \Phi$, таких что их сумма также является корнем, и для любых  $\xi \in R$, $I \trianglelefteq R$ выполнено
  $$ \left< x_\alpha(\xi) \right>^{X_\beta(I)} \ge X_{\alpha + \beta}(\xi I), $$  
  где $X_\alpha(I) = \{x_\alpha(\xi) | \xi \in I\}$ --- относительная корневая подгруппа в $G$. Запись $\left<F\right>^H$ обозначает наименьшую подгруппу, нормализуемую $H$ и содержащую $F$, то есть подгруппу, порождённую всевозможными сопряжениями $^hf$, $f \in F$, $h \in H$.
\end{lm}

\begin{lm}\label{oppositecommutator}
Если в группе Шевалле ранга не менее, чем $2$, $\beta$ --- длинный корень, то $x_\beta(\xi) \in \left<[x_{-\beta}(1),x_\beta(\xi)]\right>^E$
\end{lm}
\begin{proof}
\# Если это верно, то найти ссылку. В $\mathrm{SL}_2$ доказательство такое ($\beta=\alpha_1+\alpha_2$): 
$$
[x_{\alpha_1},[x_\beta,[x_{\alpha_2},[x_{-\beta},x_\beta]]]]=x_\beta
$$

\end{proof}

\begin{lm}\label{unipotenttransitivity}
  Любая подгруппа $H \le U$, нормализуемая $G$, имеет вид $U_I$, где $I$ --- идеал кольца.
\end{lm}
\begin{proof}
  Данное утверждение является мультипликативной записью утверждения \ref{unipotentsubgroups}.
\end{proof}


\begin{lm}\label{maxrootsum}
Максимальный корень $\beta\coloneqq\widetilde\beta_i$ можно представить в виде суммы двух корней $\gamma$ и $\delta$ одинаковой длины, так что при этом абстрактный вес $i\beta-j\gamma$ будет неотрицательным для любых положительных коэффициентов $i$ и $j$. Если же $\langle\sigma-\beta,\beta\rangle=0$ (при том, что старший вес $\sigma\coloneqq\widetilde\sigma$ не совпадает с максимальным корнем $\beta$), то выбор можно совершить таким образом, чтобы дополнительно выполнялось $\langle\sigma,\delta-\gamma\rangle>0$.

Заметим, что угол между $\gamma$ и $\delta$ в результате может оказаться либо $\frac{\pi}{2}$, либо $\frac{2\pi}{3}$. Угол $\frac{\pi}{3}$ (когда $\Phi = \mathrm{G}_2$) невозможен, так как тогда нарушится условие $i\beta-j\gamma \nprec 0$.
\end{lm}
\begin{proof}
В $\mathrm{G}_2$ требованиям леммы удовлетворяет выбор $\gamma=\alpha_2$ и $\delta=3\alpha_1+\alpha_2$.

\begin{center}
\begin{tikzpicture}[thick, scale=1]
\foreach\ang in {0,60,120}{
  \draw[blue,-{>[length=10,width=5]}] (0,0) -- ++(\ang:1);
  \draw[blue,-{>[length=10,width=5]}] (0,0) -- ++(\ang+30:{sqrt(3)});
  \draw[olive,-{>[length=10,width=5]}] (0,0) -- ++(\ang+180:1);
  \draw[olive,-{>[length=10,width=5]}] (0,0) -- ++(\ang+180+30:{sqrt(3)});
}
\path (0,0) ++(0:1) node [label=right:$\alpha_1$] {};
\path (0,0) ++(150:{sqrt(3)}) node [label=above left:{$\gamma=\alpha_2$}] {};
\path (0,0) ++(90:{sqrt(3)}) node [label=above:$\beta$] {};
\path (0,0) ++(30:{sqrt(3)}) node [label=above right:$\delta$] {};
\node[blue] at (60:1.8) {$\Phi^+$};
\node[olive] at (180+60:1.5) {$\Phi^-$};
\end{tikzpicture}
\end{center}

Пусть теперь $\Phi\ne\mathrm{G}_2$. Условие $i\beta-j\gamma\nless0\;\forall i,j\in\N$ выполняется автоматически, если угол между $\gamma$ и $\delta$ составляет $\frac{2\pi}{3}$. Если же угол $\frac{\pi}{2}$, то обеспечить выполнения этого условия можно заменой местами $\gamma$ и $\delta$, чтобы $\delta-\gamma>0$.

\begin{center}
\begin{tikzpicture}[thick, scale=1]
\foreach\ang in {0,90}{
  \draw[blue,-{>[length=10,width=5]}] (0,0) -- ++(\ang:{sqrt(2)});
  \draw[blue,-{>[length=10,width=5]}] (0,0) -- ++(\ang+45:1);
  \draw[olive,-{>[length=10,width=5]}] (0,0) -- ++(\ang+180:{sqrt(2)});
  \draw[olive,-{>[length=10,width=5]}] (0,0) -- ++(\ang+180+45:1);
}
\path (0,0) ++(0:{sqrt(2)}) node [label=right:$\delta-\gamma$] {};
\path (0,0) ++(90+45:1) node [label=above left:{$\gamma$}] {};
\path (0,0) ++(90:{sqrt(2)}) node [label=above:$\beta$] {};
\path (0,0) ++(45:1) node [label=above right:$\delta$] {};
\node[blue] at (20:1.8) {$\Phi^+$};
\node[olive] at (180+30:1.5) {$\Phi^-$};
\end{tikzpicture}
\begin{tikzpicture}[thick, scale=1]
\foreach\ang in {0,60,120}{
  \draw[blue,-{>[length=10,width=5]}] (0,0) -- ++(\ang+30:1);
  \draw[olive,-{>[length=10,width=5]}] (0,0) -- ++(\ang+180+30:1);
}
\path (0,0) ++(150:1) node [label=above left:{$\gamma$}] {};
\path (0,0) ++(90:1) node [label=above:$\beta$] {};
\path (0,0) ++(30:1) node [label=above right:$\delta$] {};
\node[blue] at (60:1) {$\Phi^+$};
\node[olive] at (180+60:1) {$\Phi^-$};
\end{tikzpicture}
\end{center}

В случае, когда $\langle\sigma-\beta,\beta\rangle=0$, обеспечить выполнения условия $\langle\sigma,\delta-\gamma\rangle>0$ можно следующим образом. Выберем корень $\delta$, не ортогональный ни $\beta$, ни $\sigma-\beta$ (это можно сделать, так как $\Phi$ неприводима). Не умаляя общности можно считать, что $\langle\delta,\beta\rangle>0$ и $\langle\delta,\sigma-\beta\rangle>0$. Тогда $\langle\delta,\beta\rangle=\frac{1}{2}\langle\beta,\beta\rangle$ и $\gamma\coloneqq\beta-\delta$. При этом
$\langle\sigma,\delta-\gamma\rangle =
\langle\sigma,2\delta-\beta\rangle =
2\langle\sigma,\delta\rangle-\langle\sigma,\beta\rangle = 
2\langle\sigma,\delta\rangle-\langle\beta,\beta\rangle =
2\langle\delta,\sigma\rangle-2\langle\delta,\beta\rangle =
2\langle\delta,\sigma-\beta\rangle > 0$.
Условие $i\beta-j\gamma\nprec\;\forall i,j\in\N$ при этом будет выполнено, так как  $\delta-\gamma$ не может быть отрицательным корнем (если это корень, то положительный).
\end{proof}

\begin{lm}\label{highestweightvariants}
При выборе $\gamma$ и $\delta$ согласно предыдущей лемме выполняется хотя бы одно из следующих условий:
\begin{itemize}
\item Если угол между $\gamma$ и $\delta$ составляет $\frac{2\pi}{3}$:
\begin{enumerate}
\item $ \sigma - \gamma \notin \Sigma$
\item $ \sigma - \delta \notin \Sigma$
\item $ \sigma - 2\gamma \in \Sigma$
\item $ \sigma - 2\delta \in \Sigma$
\end{enumerate}
\item Если угол между $\gamma$ и $\delta$ составляет $\frac{\pi}{2}$:
\begin{enumerate}
\item[0.] $ \sigma - \delta + \gamma \in \Sigma$
\item $ \sigma - 2\gamma \notin \Sigma$
\item $ \sigma - 2\delta \notin \Sigma$
\item $ \sigma - 3\gamma \in \Sigma$
\item $ \sigma - 3\delta \in \Sigma$
\end{enumerate}
\end{itemize}
\end{lm}

\begin{proof}
Очевидно что,
$$ \sigma - \frac{2 \langle \sigma,\gamma\rangle}{\langle \gamma , \gamma \rangle}\gamma \;\in\; \Sigma $$
$$ \sigma - \frac{2 \langle \sigma,\delta\rangle}{\langle \delta , \delta \rangle}\delta \;\in\; \Sigma \; , $$
но при этом
$$ \sigma - \frac{2 \langle \sigma,\gamma\rangle}{\langle \gamma , \gamma \rangle}\gamma-\gamma \;\notin\; \Sigma $$
$$ \sigma - \frac{2 \langle \sigma,\delta\rangle}{\langle \delta , \delta \rangle}\delta-\delta \;\notin\; \Sigma \; . $$
Следовательно, условия (1-4) могут выполняться, только если $\langle\sigma,\gamma\rangle=\langle\beta,\gamma\rangle$
и одновременно $\langle\sigma,\delta\rangle=\langle\beta,\delta\rangle$. Но это значит, что $\langle\sigma-\beta,\beta\rangle=0$,
а значит по предыдущей лемме $\langle\sigma,\delta-\gamma\rangle > 0$, то есть выполнено условие (0).
\end{proof}

\begin{thm}
  Снова $G = G(\Phi, R)$ --- редуктивная групповая схема Шевалле-Демазюра
  с системой корней $\Phi = \Phi_1 + \ldots + \Phi_n$, где каждая $\Phi_i$ --- неприводимая система корней ранга не меньше $2$. $U$ --- неприводимое рациональное представление $G$ (модуль Вейля), отличное от присоединённого представления группы $G$. В коммутативном кольце $R$ предполагается обратимость всех структурных констант группы $G$, а также структурных констант представления (условие \ref{representationconstantinvertibility}).
    
  Пусть имеется подгруппа $H \le G \rightthreetimes U$, нормализуемая группой $E$, то есть $[H,E] \le H$. Тогда образ $H$ при проекции $G \rightthreetimes U \rightarrow G$, обозначаемый как $H_G$, будет обладать следующим свойством: $[H_G,E]\le H$.
\end{thm}
\begin{proof}
  $H$ нормализуется $E$, следовательно $H_G$ также нормализуется $E$. По лемме \ref{directproduct} $[H_G,E] = \prod_{i=1}^n E(\Phi_i,R,I_i)$ при некотором выборе идеалов $I_i$. Требуется доказать, что $[H_G,E] \le H$, то есть в прямом произведении $\prod_{i=1}^n E(\Phi_i,R,I_i)$ каждый множитель $E(\Phi_i,R,I_i)$  лежит в $H$.
  
\begin{lm}
  $$X_{\widetilde\beta_i}(I_i)^{E(\Phi_i,R)} = E(\Phi_i,R,I_i),$$
  где $\widetilde\beta_i$ --- максимальный корень в $\Phi_i$.
\end{lm}
\begin{proof}
  Очевидно, что
\begin{align*}
  X_{\widetilde\beta_i}(I_i) &\le E(\Phi_i,I_i) \\
  X_{\widetilde\beta_i}(I_i)^{E(\Phi_i,R)} &\le E(\Phi_i,I_i)^{E(\Phi_i,R)} = E(\Phi_i,R,I_i)
\end{align*}
  Обратное включение вытекает из леммы \ref{transitivity}.
  
  Действительно, возьмём в лемме \ref{transitivity} в качестве $\alpha$ максимальный корень $\widetilde{\beta_i}$, а в качестве идеала $I$ всё кольцо $R$. Тогда
  $$ \left< x_{\widetilde\beta_i}(\xi) \right>^{X_\beta(R)} \ge X_{\widetilde\beta_i + \beta}(\xi R). $$
  Но так как любой корень из $\Phi_i$ может быть получен прибавлением к максимальному корню $\widetilde\beta_i$ некоторого отрицательного корня $\beta \in \Phi^-$, то группа $E(\Phi_i,I_i)$ порождается подгруппами $X_{\widetilde\beta_i + \beta}(I_i)$, а следовательно 
\begin{align*}
E(\Phi_i,I_i) &\le \left< X_{\widetilde\beta_i}(I) \right>^{E(\Phi_i,R)}\\
  E(\Phi_i,R,I_i) = E(\Phi_i,I_i)^{E(\Phi_i,R)} &\le \left< X_{\widetilde\beta_i}(I) \right>^{E(\Phi_i,R)}
\end{align*}
\end{proof}

Таким образом, необходимо доказать, что $X_{\widetilde\beta_i}(I_i)^{E(\Phi_i,R)} \le H$, то есть что
$x_{\widetilde{\beta_i}}(\xi) \in H \ \forall \xi \in I_i$.

По построению известно, что $x_{\widetilde\beta_i}(\xi) \in [H_G,E] \le H_G$.
Обозначим за $h$ некоторый прообраз $x_{\widetilde\beta_i}(\xi)$ при проекции $G \rightthreetimes U \rightarrow G$.

\begin{equation*}
\tikzset{
  Subgroup/.style={
    draw=none,
    every to/.append style={
      edge node={node [sloped, allow upside down, auto=false]{$\le$}}}},
  Equals/.style={
    draw=none,
    every to/.append style={
      edge node={node [sloped, allow upside down, auto=false]{$=$}}}},
  Included/.style={
    draw=none,
    every to/.append style={
      edge node={node [sloped, allow upside down, auto=false]{$\in$}}}}
}
\begin{tikzcd}
G \rightthreetimes U \arrow{r}{} & G \\
H \arrow[Subgroup]{u} \arrow{r}{} & H_G \arrow[Subgroup]{u} \\
h \arrow[Included]{u} \arrow[maps to]{r}{} & x_{\widetilde\beta_i}(\xi) \arrow[Included]{u} \\
x_{\widetilde\beta_i}(\xi)\,u \arrow[Equals]{u} \arrow[maps to]{r}{} & x_{\widetilde\beta_i}(\xi) \arrow[Equals]{u} \\
\end{tikzcd}
\end{equation*}

Рассмотрим два случая. Пусть сначала $u \notin U_{\widetilde\sigma}$, то есть $u$ не является вектором максимального веса. Разложим $u$ по весовым подпространствам: $$u = \prod_{\sigma \in \Sigma} u_\sigma = u_{\widetilde\sigma} \, \prod_{\sigma \in \Sigma \setminus \{\widetilde\sigma\} } u_\sigma = u_{\widetilde\sigma} \, u'  .$$
Если $u_\sigma$ --- один из множителей, составляющих $u'$, то найдётся положительный корень $\beta \in \Phi_i^+$, такой что $\beta + \sigma \in \Sigma$, а значит, $[x_\beta(1), u] \ne 1$.

Вычислим
\begin{multline*}
[x_\beta(1), h] = [x_\beta(1), x_{\widetilde\beta_i}(\xi)u] =
{}^{x_\beta(1)}(x_{\widetilde\beta_i}(\xi)u)\,u^{-1} {x_{\widetilde\beta_i}(\xi)}^{-1}
= \\ =
{}^{x_\beta(1)}x_{\widetilde\beta_i}(\xi) \, {}^{x_\beta(1)}u\,u^{-1} {x_{\widetilde\beta_i}(\xi)}^{-1} =
{}^{x_{\widetilde\beta_i}(\xi)}[x_\beta(1),u] \; \in \; U \cap H \; .
\end{multline*}

Тогда 

$$ w \coloneqq {}^{x_{\widetilde\beta_i}(\xi)^{-1}} [x_\beta(1), h] = [x_\beta(1),u] \; \in \; U \cap H \; .$$

Следовательно, так как $(U \cap H)^E = U \cap H$, то по свойству \ref{unipotentsubgroups}, $u \; \in \; U \cap H$, а следовательно, $x_{\widetilde\beta_i(\xi)} \in H$ (\textbf{\color{red}ЧУШЬ!!!}).

Пусть теперь $u \in  U_{\widetilde\sigma}$. 

Представим $\widetilde\beta$ в виде суммы $\gamma+\delta$ в соответствии с леммой \ref{maxrootsum}.

Рассмотрим случаи в соответствии с леммой \ref{highestweightvariants}.
Если $\sigma-\delta+\gamma \in \Sigma$ (случай (0)), то
$$ [x_{\gamma-\delta}(1),h] = [x_{\gamma-\delta}(1),x_\beta(\xi)u] = [x_{\gamma-\delta}(1),x_\beta(\xi)] \cdot {}^{x_\beta(\xi)}[x_{\gamma-\delta}(1),u] = {}^{x_\beta(\xi)}[x_{\gamma-\delta}(1),u] \;\in\; U\cap H $$

Обозначим
\begin{multline*}
h_{-\gamma} \coloneqq [x_{-\gamma}(1),h] = [x_{-\gamma}(1),x_\beta(\xi) u] = [x_{-\gamma}(1),x_\beta(\xi)] \cdot {}^{x_\beta(\xi)}[x_{-\gamma}(1),u] = \\ =
x_{\beta-\gamma}(N_{-\gamma,\beta,1,1} \,\xi) \, x_{\beta-2\gamma}(\ldots) \, \varepsilon_{-\gamma}(u) \, u_1 \; ,
\end{multline*}
где $u_1$ раскладывается по подпространствам $U_\sigma$ с $\sigma \in \widetilde\sigma-\gamma - \N \, \gamma$ (абстрактные веса $\{\widetilde\sigma-\N\,\gamma+\N\,\beta\}$ не могут являться весами представления, так как среди $\{\N\,\beta-\N\,\gamma\}$ нет отрицательных абстрактных весов).

\begin{center}
\begin{tikzpicture}[thick, scale=1]
\newcommand{\point}[1]{node (#1) [circle,inner sep=1,fill] {}}
\draw[dotted] (0,0) node (sigma) [circle,inner sep=2,fill=blue,label=above right:{$\widetilde\sigma$}] {} ++(-45:1) node (sigmaminusgamma) [circle,inner sep=2,fill=blue,label=above right:{$\widetilde\sigma-\gamma$}] {} -- ++(-90-45:1) \point{} -- ++(90+45:1) \point{} -- +(-90-45:0.8) ++(-45:1) -- +(-90-45:0.8) ++(0:0) -- ++(-45:1) \point{} -- ++(45:1) \point{} -- +(-45:0.8) ++(-90-45:1) -- +(-45:0.8) ++(0:0) -- +(-90-45:0.8) ++(45:1) -- (sigmaminusgamma) (sigma) -- +(-90-45:1);
\draw[blue,-{>[length=10,width=5]}] (sigma) -- (sigmaminusgamma);
\draw[red,pattern=north west lines, pattern color=red] (-45:2.8) ++(45:0.3) -- ++(90+45:1.3) to[bend right=90] ++(-90-45:0.6) -- ++(-45:1.3) ++(45:0.3);
\node[red] at (-45:3.0) {$u_1$};
\end{tikzpicture}
\end{center}

\begin{multline*}
h_{-2\gamma} \coloneqq [x_{-\gamma}(1),h_{-\gamma}] = \\ =
[x_{-\gamma}(1), x_{\beta-\gamma}(N_{-\gamma,\beta,1,1} \,\xi) \, x_{\beta-2\gamma}(\ldots) \, \varepsilon_{-\gamma}(u) \, u_1] = \\ =
[x_{-\gamma}(1), x_{\beta-\gamma}(N_{-\gamma,\beta,1,1} \,\xi) \, x_{\beta-2\gamma}(\ldots)] \cdot {}^{x_{\beta-\gamma}(N_{-\gamma,\beta,1,1} \,\xi) \, x_{\beta-2\gamma}(\ldots)} [x_{-\gamma}(1), \varepsilon_{-\gamma}(u) \, u_1] = \\ =
x_{\beta-2\gamma}(N_{-\gamma,\beta,1,1} N_{-\gamma,\beta-\gamma,1,1} \, \xi) \; \varepsilon^2_{-\gamma}(u) \, u_2 \; ,
\end{multline*}
где $u_2$ раскладывается по подпространствам $U_\sigma$ с $\sigma \in \widetilde\sigma-2\gamma - \N \, \gamma$.

\begin{center}
\begin{tikzpicture}[thick, scale=1]
\newcommand{\point}[1]{node (#1) [circle,inner sep=1,fill] {}}
\draw[dotted] (0,0) node (sigma) [circle,inner sep=2,fill=blue,label=above right:{$\widetilde\sigma$}] {} ++(-45:1) node (sigmaminusgamma) [circle,inner sep=2,fill=blue,label=above right:{$\widetilde\sigma-\gamma$}] {} ++(-45:1) node (sigmaminus2gamma) [circle,inner sep=2,fill=blue,label=above right:{$\widetilde\sigma-2\gamma$}] {} (sigmaminusgamma) -- ++(-90-45:1) \point{} -- ++(90+45:1) \point{} -- +(-90-45:0.8) ++(-45:1) -- +(-90-45:0.8) ++(0:0) -- ++(-45:1) \point{} -- (sigmaminus2gamma) -- +(-45:0.8) ++(-90-45:1) -- +(-45:0.8) ++(0:0) -- +(-90-45:0.8) (sigma) -- +(-90-45:1);
\draw[blue,-{>[length=10,width=5]}] (sigma) -- (sigmaminusgamma);
\draw[blue,-{>[length=10,width=5]}] (sigmaminusgamma) -- (sigmaminus2gamma);
\draw[red,pattern=north west lines, pattern color=red] (-45:2.8) ++(45:0.3) -- ++(90+45:0.3) to[bend right=90] ++(-90-45:0.6) -- ++(-45:0.3) ++(45:0.3);
\node[red] at (-45:3.0) {$u_2$};
\end{tikzpicture}
\end{center}

Если угол между $\gamma$ и $\delta$ составляет $\frac{\pi}{2}$, и при этом $\widetilde\sigma-3\gamma\in\Sigma$, то
\begin{multline*}
h_{-3\gamma} \coloneqq [x_{-\gamma}(1),h_{-2\gamma}] = \\ =
[x_{-\gamma}(1), x_{\beta-2\gamma}(N_{-\gamma,\beta,1,1} N_{-\gamma,\beta-\gamma,1,1} \, \xi) \; \varepsilon^2_{-\gamma}(u) \, u_2] = \\ =
{}^{x_{\beta-2\gamma}(N_{-\gamma,\beta,1,1} N_{-\gamma,\beta-\gamma,1,1} \, \xi)}[x_{-\gamma}(1),\varepsilon^2_{-\gamma}(u) \, u_2] =
\varepsilon^3_{-\gamma}(u) \, u_3
 \; ,
\end{multline*}
где $u_3$ раскладывается по подпространствам $U_\sigma$ с $\sigma \in \widetilde\sigma-3\gamma - \N \, \gamma$.
При этом $h_{-3\gamma}\;\in\;U\cap H$, а их свойства \ref{weightprojections} вытекает, что $\varepsilon^3_{-\gamma}(u)$ также лежит в $H$. \#И тут нужно сделать вывод, что и $u$ лежит в $H$.

Если угол между $\gamma$ и $\delta$ составляет $\frac{2\pi}{3}$, и при этом $\widetilde\sigma-2\gamma\in\Sigma$, то $h_{-2\gamma}\;\in\;U\cap H$ с аналогичными выводами.

Пусть теперь угол между $\gamma$ и $\delta$ составляет $\frac{\pi}{2}$, и $\widetilde\sigma-3\gamma\notin\Sigma$. В этом случае $u_2=1$ и
\begin{multline*}
h_{-\gamma-\beta} \coloneqq [x_{-\delta}(1),h_{-2\gamma}] = \\ =
[x_{-\delta}(1), x_{\beta-2\gamma}(N_{-\gamma,\beta,1,1} N_{-\gamma,\beta-\gamma,1,1} \, \xi) \; \varepsilon^2_{-\gamma}(u)] = \\ =
[x_{-\delta}(1), x_{\beta-2\gamma}(N_{-\gamma,\beta,1,1} N_{-\gamma,\beta-\gamma,1,1} \, \xi)] \cdot {}^{x_{\beta-2\gamma}(N_{-\gamma,\beta,1,1} N_{-\gamma,\beta-\gamma,1,1} \, \xi)}[x_{-\delta}(1),\varepsilon^2_{-\gamma}(u)] = \\ =
x_{-\gamma}(N_{-\gamma,\beta,1,1} N_{-\gamma,\beta-\gamma,1,1} N_{-\delta,\beta-2\gamma,1,1} \, \xi) x_{-\beta}(\ldots) \; \varepsilon_{-\delta}(\varepsilon_{-\gamma}^2(u))\,u_4
 \; ,
\end{multline*}
где $u_4$ раскладывается по подпространствам $U_\sigma$ с $\sigma \in \widetilde\sigma-\gamma-\beta- \N\delta + \N_0 (\gamma-\delta)$.

\begin{center}
\begin{tikzpicture}[thick, scale=1]
\newcommand{\point}[1]{node (#1) [circle,inner sep=1,fill] {}}
\draw[dotted] (0,0) node (sigma) [circle,inner sep=2,fill=blue,label=above right:{$\widetilde\sigma$}] {} ++(-45:1) node (sigmaminusgamma) [circle,inner sep=2,fill=blue,label=above right:{$\widetilde\sigma-\gamma$}] {} ++(-45:1) node (sigmaminus2gamma) [circle,inner sep=2,fill=blue,label=above right:{$\widetilde\sigma-2\gamma$}] {} ++(-90-45:1) node (sigmaminusgammabeta) [circle,inner sep=2,fill=blue] {} node[above left,fill=white,inner sep=0.5pt,rounded corners=6pt]{$\widetilde\sigma-\gamma-\beta$} (sigmaminusgamma) -- ++(-90-45:1) \point{} -- ++(90+45:1) \point{} -- +(-90-45:0.3) ++(-45:1) -- +(-90-45:1) \point{} ++(0:0) -- (sigmaminusgammabeta) -- ++(-45:1) \point{} -- +(-90-45:0.6) +(0:0) -- (sigmaminus2gamma) ++(-90:{sqrt(2)}) -- +(-90:{0.6/sqrt(2)}) (sigmaminusgammabeta) -- ++(-90-45:1) \point{} -- +(-90-45:0.6) ++(0:0) -- +(-45:0.6) ++(0:0) -- ++(90+45:1) -- +(-90-45:0.6) ++(0:0) -- +(90+45:0.5) (sigma) -- +(-90-45:1);
\draw[blue,-{>[length=10,width=5]}] (sigma) -- (sigmaminusgamma);
\draw[blue,-{>[length=10,width=5]}] (sigmaminusgamma) -- (sigmaminus2gamma);
\draw[blue,-{>[length=10,width=5]}] (sigmaminus2gamma) -- (sigmaminusgammabeta);
\path[fill,apply={draw} except on segments {1},red,pattern=north west lines, pattern color=red] ({sqrt(2)+0.3},{-2.3*sqrt(2)}) -- ({sqrt(2)+0.3},{-2*sqrt(2)+0.3}) -- +(-180:{sqrt(2)+0.3}) to[bend right=22.5] ({-0.3/sqrt(2)},{-2*sqrt(2)+0.3/sqrt(2)}) -- +(-90-45:0.9) ({sqrt(2)+0.3},{-2.3*sqrt(2)});
\node[red] at ({0.5,-3.5}) {$u_4$};
\end{tikzpicture}
\end{center}

А если угол между $\gamma$ и $\delta$ составляет $\frac{2\pi}{3}$, и $\widetilde\sigma-2\gamma\notin\Sigma$, то $u_1=1$ и
\begin{multline*}
h_{-\gamma-\beta} \coloneqq [x_{-\beta}(1),h_{-\gamma}] = \\ =
[x_{-\beta}(1), x_{\beta-\gamma}(N_{-\gamma,\beta,1,1} \,\xi) \, \varepsilon_{-\gamma}(u)] = \\ =
[x_{-\beta}(1), x_{\beta-\gamma}(N_{-\gamma,\beta,1,1} \,\xi)] \cdot {}^{x_{\beta-\gamma}(N_{-\gamma,\beta,1,1} \,\xi)}[x_{-\beta}(1),\varepsilon_{-\gamma}(u)] = \\ =
x_{-\gamma}(N_{-\gamma,\beta,1,1} N_{-\beta,\beta-\gamma,1,1} \, \xi) \; \varepsilon_{-\beta}(\varepsilon_{-\gamma}(u))\,u_4
 \; ,
\end{multline*}
где $u_4$ раскладывается по подпространствам $U_\sigma$ с $\sigma \in \widetilde\sigma-\gamma-\beta - i \beta - j \gamma$, $i,j\in \N_0$, $i+j>0$.

\begin{center}
\begin{tikzpicture}[thick, scale=1]
\newcommand{\point}[1]{node (#1) [circle,inner sep=1,fill] {}}
\draw[dotted] (0,0) node (sigma) [circle,inner sep=2,fill=blue,label=above right:{$\widetilde\sigma$}] {} ++(-30:1) node (sigmaminusgamma) [circle,inner sep=2,fill=blue,label=above right:{$\widetilde\sigma-\gamma$}] {} ++(-30:1) ++(-150:1) node (sigmaminusgammabeta) [circle,inner sep=2,fill=blue,label=above right:{$\widetilde\sigma-\gamma-\beta$}] {} (sigmaminusgamma) -- ++(-150:1) \point{} -- ++(150:1) \point{} -- +(-150:0.3) ++(0:0) -- (sigma) -- ++(-90:1) -- +(-90:1) ++(0:0) -- +(-150:0.6) ++(0:0) -- (sigmaminusgammabeta) -- ++(-30:1) \point{} -- +(-30:0.4) ++(0:0) -- +(-90:0.4) ++(0:0) -- +(-150:1) (sigmaminusgammabeta) -- +(-150:1) (sigmaminusgammabeta) -- ++(-90:1) \point{} -- +(-90:0.4) ++(0:0) -- +(-30:0.4) ++(0:0) -- +(-150:0.6) ++(0:0) -- ++(150:1) \point{} -- +(-90:0.6) ++(0:0) -- +(-150:0.6) ++(0:0) -- +(150:0.6);
\draw[blue,-{>[length=10,width=5]}] (sigma) -- (sigmaminusgamma);
\draw[blue,-{>[length=10,width=5]}] (sigmaminusgamma) -- (sigmaminusgammabeta);
\draw[fill,red,pattern=north west lines, pattern color=red] (sigmaminusgammabeta) ++(-30:1.6) ++(60:0.3) -- ++(150:0.6) to[bend right=30] ++(-180:0.3) -- ++(-150:1) to[bend right=30] ++(-120:0.3) -- ++(-90:0.6);
\node[red] at ({1.5,-3}) {$u_4$};
\end{tikzpicture}
\end{center}

Таким образом, в обоих случаях имеет место
$$
h_{-\gamma-\beta} =
x_{-\gamma}(N\xi) x_{-\beta}(\ldots) \; u_{-\gamma-\beta} \,u_4
 \; ,
$$
где $$
N\coloneqq 
\begin{cases}
  N_{-\gamma,\beta,1,1} N_{-\gamma,\beta-\gamma,1,1} N_{-\delta,\beta-2\gamma,1,1},& \text{если угол между } \gamma \text{ и } \delta \text{ составляет } \frac{\pi}{2}\\
  N_{-\gamma,\beta,1,1} N_{-\beta,\beta-\gamma,1,1},              & \text{если угол между } \gamma \text{ и } \delta \text{ составляет } \frac{2\pi}{3}\\
\end{cases}
$$
$$
u_{-\gamma-\beta}\coloneqq 
\begin{cases}
  \varepsilon_{-\delta}(\varepsilon_{-\gamma}^2(u)),& \text{если угол между } \gamma \text{ и } \delta \text{ составляет } \frac{\pi}{2}\\
  \varepsilon_{-\beta}(\varepsilon_{-\gamma}(u)),              & \text{если угол между } \gamma \text{ и } \delta \text{ составляет } \frac{2\pi}{3}\\
\end{cases}
$$
При этом $u_4$ в обоих случаях раскладывается по подпространствам $U_\sigma$ с $\sigma \in \widetilde\sigma-\gamma-\beta- \N\delta + \N_0 (\gamma-\delta)$.

\begin{center}
\begin{tikzpicture}[thick, scale=1]
\newcommand{\point}[1]{node (#1) [circle,inner sep=1,fill] {}}
\draw[dotted] ({1/sqrt(2)},{-3/sqrt(2)}) node (sigmaminusgammabeta) [circle,inner sep=2,fill=blue] {} node[above left,fill=white,inner sep=0.5pt,rounded corners=6pt]{$\widetilde\sigma-\gamma-\beta$} -- ++(-45:1) \point{} -- +(-45:0.6) ++(0:0) -- +(-90-45:0.6) ++(0:0) -- +(-90:{0.6/sqrt(2)}) (sigmaminusgammabeta) -- ++(-90-45:1) \point {} -- +(-90-45:0.6) ++(0:0) -- +(-45:0.6) ++(0:0) -- +(-90:{0.6/sqrt(2)}) (sigmaminusgammabeta) -- +(-90:{1.6/sqrt(2)});
\path[fill,apply={draw} except on segments {1},red,pattern=north west lines, pattern color=red] ({sqrt(2)+0.7},{-2.3*sqrt(2)}) -- ({sqrt(2)+0.7},{-2*sqrt(2)+0.3}) -- +(-180:{sqrt(2)+0.7}) to[bend right=22.5] ({-0.3/sqrt(2)},{-2*sqrt(2)+0.3/sqrt(2)}) -- +(-90-45:0.9) ({sqrt(2)+0.3},{-2.3*sqrt(2)});
\node[red] at ({0.5,-3.5}) {$u_4$};
\end{tikzpicture}
\end{center}

После этого вычислим
\begin{multline*}
h_{-2\beta} \coloneqq [x_{-\delta}(1),h_{-\gamma-\beta}] = \\ =
[x_{-\delta}(1), x_{-\gamma}(N\xi)x_{-\beta}(\ldots) \; u_{-\gamma-\beta} \,u_4] = \\ =
[x_{-\delta}(1), x_{-\gamma}(N\xi)x_{-\beta}(\ldots)] \cdot {}^{x_{-\gamma}(N\xi)x_{-\beta}(\ldots)}[x_{-\delta}(1),u_{-\gamma-\beta} \,u_4] = \\ =
x_{-\beta}(N \, N_{-\delta,-\gamma,1,1} \, \xi) \; \varepsilon_{-\delta}(u)\,u_5
 \; ,
\end{multline*}
где $u_5$ раскладывается по подпространствам $U_\sigma$ с $\sigma \in \widetilde\sigma-2\beta - \N\delta + \N_0 \, (\delta-\gamma)$.

\begin{center}
\begin{tikzpicture}[thick, scale=1]
\newcommand{\point}[1]{node (#1) [circle,inner sep=1,fill] {}}
\draw[dotted] (0,{-2*sqrt(2)}) node (sigmaminus2beta) [circle,inner sep=2,fill=blue] {} node[above left,fill=white,inner sep=0.5pt,rounded corners=6pt]{$\widetilde\sigma-2\beta$} -- ++(-45:1) \point{} -- +(-45:0.6) ++(0:0) -- +(-90-45:0.6) ++(0:0) -- +(-90:{0.6/sqrt(2)}) (sigmaminus2beta) -- ++(-90-45:1) \point {} -- +(-90-45:0.6) ++(0:0) -- +(-45:0.6) ++(0:0) -- +(-90:{0.6/sqrt(2)}) (sigmaminus2beta) -- +(-90:{1.6/sqrt(2)});
\path[fill,apply={draw} except on segments {1},red,pattern=north west lines, pattern color=red] ({sqrt(2)/2+0.7},{-2.8*sqrt(2)}) -- ({sqrt(2)/2+0.7},{-2.5*sqrt(2)+0.3}) -- +(-180:{sqrt(2)+0.7}) to[bend right=22.5] ({-1.3/sqrt(2)},{-2*sqrt(2)-0.7/sqrt(2)}) -- +(-90-45:0.9) ({sqrt(2)/2+0.3},{-2.8*sqrt(2)});
\node[red] at ({-0.2,-4.2}) {$u_4$};
\end{tikzpicture}
\end{center}

Если $\langle\widetilde\sigma,\beta\rangle<0$, то $\widetilde\sigma-2\beta\notin\Sigma$, а следовательно
$$h_{-2\beta} = x_{-\beta}(N \, N_{-\delta,-\gamma,1,1} \, \xi) \; \in \; E \cap H \;.$$

Если $\langle\widetilde\sigma,\beta\rangle>0$, то $\widetilde\sigma-3\beta\in\Sigma$. Тогда вычислим
\begin{multline*}
h_{-3\beta} \coloneqq [x_{-\beta}(1),h_{-2\beta}] = \\ =
[x_{-\beta}(1), x_{-\beta}(N \, N_{-\delta,-\gamma,1,1} \, \xi) \; \varepsilon_{-\delta}(u)\,u_5] = \\ =
[x_{-\beta}(1), x_{-\beta}(N \, N_{-\delta,-\gamma,1,1} \, \xi)] \cdot {}^{x_{-\beta}(N \, N_{-\delta,-\gamma,1,1} \, \xi)}[x_{-\beta}(1),\varepsilon_{-\delta}(u)\,u_5] = \\ =
\varepsilon_{-\beta}(\varepsilon_{-\delta}(u))\,u_6 \; \in \; U \cap H
 \; ,
\end{multline*}
где $u_6$ раскладывается по подпространствам $U_\sigma$ с $\sigma \in \widetilde\sigma-3\beta - \N\delta + \N_0 \, (\delta-\gamma)$. \#Здесь снова нужно сделать вывод, что $u\in H$.

Остался случай, когда $\langle\widetilde\sigma,\beta\rangle=0$ и $\langle\sigma,\delta-\gamma\rangle>0$. Докажем, что в этом случае 
$$
\begin{cases}
  \sigma-2\gamma\notin\Sigma,& \text{если угол между } \gamma \text{ и } \delta \text{ составляет } \frac{\pi}{2}\\
  \sigma-\gamma\notin\Sigma,              & \text{если угол между } \gamma \text{ и } \delta \text{ составляет } \frac{2\pi}{3}\\
\end{cases}$$

\#Почему это должно быть верно:

\begin{center}
\newcommand{\point}[1]{node (#1) [circle,inner sep=1,fill] {}}
\newcommand{\pointroot}[1]{node (#1) [circle,inner sep=2,fill=blue] {}}
\newcommand{\pointweight}[1]{node (#1) [circle,inner sep=2,fill=olive] {}}
\newcommand{\pointnotweight}[1]{node (#1) [circle,inner sep=2,fill=red] {}}

\begin{tikzpicture}[thick, scale=1]

\path (0,0) \pointroot{zero};

\foreach\ang in {0,60,...,300}{
  \draw[dotted] (zero) ++(\ang:1) \pointroot{root} -- ++(\ang:1) \point{} -- ++(\ang+120:1) \point{} -- ++(\ang+240:1) -- ++(\ang+120:1) -- ++(\ang:1) \point{} -- ++(\ang+120:1);
  \draw[->,dotted,blue] (zero) -- (root);
}

\path (zero) ++(30:{1/sqrt(3)}) \pointweight{pi1};

\draw (pi1) ++(240:1) \pointweight{} ++(180:1) \pointweight{} ++(0:2) \pointweight{} ++(120:2) \pointweight{sigma} ++(240:1) \pointweight{sigmaminusbeta} ++(120:1) \pointnotweight{notweight1} ++(0:2) \pointnotweight{notweight2};

\draw[->,blue] (sigma) -- (sigmaminusbeta);
\draw[->,blue] (sigmaminusbeta) -- (notweight1);
\draw[->,olive] (notweight1) edge[bend left=40] (notweight2);

\end{tikzpicture}

\begin{tikzpicture}[thick, scale=1]

\path (0,0) \pointroot{zero};

\foreach\ang in {0,1,...,3}{
  \draw[dotted] (zero) ++(90*\ang-45:1) \pointroot{short\ang} -- ++(90*\ang+45:1) \pointroot{long\ang} -- ++(90*\ang+135:1) -- ++(90*\ang+45:1) \point{} -- ++(90*\ang-45:1) \point{} -- ++(90*\ang-135:1) -- ++(90*\ang-45:1) \point{} +(90*\ang-135:1) -- ++(90*\ang+45:1) \point{} -- ++(90*\ang+135:1);
  \draw[->,dotted,blue] (zero) -- (short\ang);
  \draw[->,dotted,blue] (zero) -- (long\ang);
  
  \path (zero) -- ++(90*\ang:{1/sqrt(2)}) \pointweight{sigmaminusbeta\ang} -- ++(90*\ang+45:1) \pointweight{} -- ++(90*\ang+135:1) \pointweight{sigma\ang};
}

\draw (sigma0) +(-45:2) \pointnotweight{notweight1} +(135:1) \pointnotweight{notweight2};

\draw[->,blue] (sigma0.south east) -- (sigmaminusbeta0.north east);
\draw[->,blue] (sigmaminusbeta0.south east) -- (notweight1.south west);
\draw[->,olive] (notweight1) edge[bend left=-20] (notweight2);

\end{tikzpicture}

\end{center}

\end{proof}
\bibliographystyle{plain}
\bibliography{document}
\end{document}
